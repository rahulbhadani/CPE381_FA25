%\documentclass[11pt, xcolor=dvipsnames,svgnames,x11names]{article}
\documentclass[nohyper,nobib,xcolor=dvipsnames,svgnames,x11names]{tufte-book}

\title{Signals and Systems for Computer Engineers}
\date{Worksheet}
\author[Rahul Bhadani]{Rahul Bhadani}
\publisher{Department of Electrical \& Computer Engineering, The University of Alabama in Huntsville}


%%%%%%%%%%%%%%%%%%%%%%%%%%%%%%%%%%%%%%%%%%
\usepackage{hyperref}
\hypersetup{
    colorlinks=true,
    linkcolor=SeaGreen,
    urlcolor=SeaGreen,
    citecolor=SeaGreen,
    filecolor=SeaGreen
}
\usepackage{amsmath}
\usepackage{amssymb}
\usepackage{xcolor}
\usepackage{tikz}
\usetikzlibrary{backgrounds}
\usetikzlibrary{shadings}
\usepackage{eso-pic}
\usepackage{tkz-euclide}
\usepackage{graphicx}
\usepackage[T1]{fontenc}
\usepackage{wesa}


\newenvironment{multiequation}{%
  \setlength{\abovedisplayskip}{3pt}      % Reduce space above the equation
  \setlength{\belowdisplayskip}{6pt}      % Reduce space below the equation
  \setlength{\abovedisplayshortskip}{3pt} % Reduce space above short equations
  \setlength{\belowdisplayshortskip}{6pt} % Reduce space below short equations
  \begin{equation}%                       % Start an unnumbered equation
    \begin{aligned}%                      % Align multiple lines
}{%
    \end{aligned}%                        % End alignment
  \end{equation}%                         % End equation
}


\definecolor{darkrose}{HTML}{9F3450}
\definecolor{forestgreen}{HTML}{35a17f}
\definecolor{lightindigo}{RGB}{150, 150, 255}
\definecolor{faintred}{RGB}{255, 200, 200}


%%%%%%%%%%%%%%%%%%%%%%%%%%%%%%%%%%%%%%%%%%%
\makeatletter
\renewcommand{\maketitlepage}{%
\begingroup%
\setlength{\parindent}{0pt}

{\fontsize{24}{24}\selectfont\textit{\@author}\par}

\vspace{1.5in}{\fontsize{30}{24}\selectfont\@title\par}


\vspace{0.5in}{\fontsize{20}{14}\selectfont\textsf{\smallcaps{\@date}}\par}




\vfill{\fontsize{14}{14}\selectfont\textit{\@publisher}\par}

\thispagestyle{empty}
\endgroup
}
\makeatother



\titlecontents{part}%
    [0pt]% distance from left margin
    {\addvspace{0.25\baselineskip}}% above (global formatting of entry)
    {\allcaps{Part~\thecontentslabel}\allcaps}% before w/ label (label = ``Part I'')
    {\allcaps{Part~\thecontentslabel}\allcaps}% before w/o label
    {}% filler and page (leaders and page num)
    [\vspace*{0.5\baselineskip}]% after

\titlecontents{chapter}%
    [4em]% distance from left margin
    {}% above (global formatting of entry)
    {\contentslabel{2em}\textit}% before w/ label (label = ``Chapter 1'')
    {\hspace{0em}\textit}% before w/o label
    {\qquad\thecontentspage}% filler and page (leaders and page num)
    [\vspace*{0.5\baselineskip}]% after
%%%% End additional code by Kevin Godby

\makeatletter
% Redefine chapter to prevent empty pages
\renewcommand{\chapter}{%
  \clearpage% Use clearpage instead of cleardoublepage
  \thispagestyle{plain}%
  \global\@topnum\z@
  \@afterindentfalse
  \secdef\@chapter\@schapter
}

% Alternative approach - modify the openright behavior
\@openrightfalse

% Or if the above doesn't work, try this more direct approach:
\let\cleardoublepage\clearpage
\makeatother



\makeatletter
% Force chapter numbering
\setcounter{secnumdepth}{2}

% Override tufte-book's chapter command to include numbering
\let\oldchapter\chapter
\renewcommand{\chapter}[1]{%
  \refstepcounter{chapter}%
  \oldchapter*{\color{darkrose}{\thechapter. #1}}% Use chapter* with our numbering
  \addcontentsline{toc}{chapter}{\protect\numberline{\thechapter}#1}%
}

% Handle unnumbered chapters (like those created with \chapter*)
\let\oldschapter\@schapter
\renewcommand{\@schapter}[1]{%
  \oldschapter{#1}%
}
\makeatother


\begin{document}
\maketitle


\chapter{Mathematical Preliminaries}



\section{Complex Numbers}

\begin{enumerate}
    \item \textbf{4pts} Plot the following complex numbers on the complex plane:
    \begin{itemize}
        \item $2+i3$
        \item $3-i2$
        \item $-2-i2$
        \item $-4+j2$
    \end{itemize}


\textbf{Solution:}

\includegraphics[width=1.0\linewidth]{../Code/figures/Ch01_complex_number_plotting.pdf}


    \item \textbf{4pts} Express $\frac{-1+3i}{2+5i}$ in the form $a+ib$.

\textbf{Solution:}    
    
    \begin{multiequation}
        \frac{-1+3i}{2+5i} &= \frac{-1+3i}{2+5i} \times \frac{2-5i}{2-5i} \\
        &= \frac{-2+5i+6i-15i^2}{2^2+5^2} \\
        &= \frac{-2+11i+15}{4+25} \\
        &= \frac{13+11i}{29} \\
        &= \frac{13}{29} + \frac{11}{29}i
    \end{multiequation}
    

    \item \textbf{2pts} If $Z=3+i5$ is a complex number, what is the value of the modulus $|Z|$?
    
    \textbf{Solution:}
    
    $$|Z| = \sqrt{3^2+5^2} = \sqrt{9+25} = \sqrt{34}$$

    \item \textbf{4pts} Find the roots of the equation $x^2+x+1=0$.

\textbf{Solution:}    
    
    For a quadratic equation $ax^2+bx+c=0$, the roots are given by $x=\frac{-b\pm\sqrt{b^2-4ac}}{2a}$.
    In this case, $a=1$, $b=1$, $c=1$.
    $$x = \frac{-1\pm\sqrt{1^2-4(1)(1)}}{2(1)} = \frac{-1\pm\sqrt{1-4}}{2} = \frac{-1\pm\sqrt{-3}}{2} = \frac{-1\pm i\sqrt{3}}{2}$$

    \item \textbf{2pts} Write the following complex numbers in the polar form:
    \begin{enumerate}
        \item $z=1+i$
        \item $w=\sqrt{3}-i$
    \end{enumerate}
    \textbf{Solution for (a):}
    $|z|=\sqrt{1^2+1^2}=\sqrt{2}$
    $\tan\theta = \frac{1}{1}=1$, so $\theta = \frac{\pi}{4}$.
    In polar form: $z=\sqrt{2}(\cos\frac{\pi}{4}+i\sin\frac{\pi}{4})$.

    \textbf{Solution for (b):}
    $|w|=\sqrt{(\sqrt{3})^2+(-1)^2}=\sqrt{3+1}=\sqrt{4}=2$
    $\tan\theta = \frac{-1}{\sqrt{3}}$, so $\theta = -\frac{\pi}{6}$ (since the number is in the 4th quadrant).
    In polar form: $w=2(\cos(-\frac{\pi}{6})+i\sin(-\frac{\pi}{6}))$.

    \item \textbf{4pts} Find the product of the complex numbers $1+i$ and $\sqrt{3}-i$ in the polar form.
    
    \textbf{Solution:}
    
    From the previous problem, we have:
    $z_1 = 1+i = \sqrt{2}e^{i\pi/4}$
    
    $z_2 = \sqrt{3}-i = 2e^{-i\pi/6}$
    
    $z_1z_2 = (\sqrt{2}e^{i\pi/4})(2e^{-i\pi/6}) = 2\sqrt{2}e^{i(\pi/4-\pi/6)} = 2\sqrt{2}e^{i(\pi/12)}$
    
    In standard polar form: $2\sqrt{2}(\cos\frac{\pi}{12}+i\sin\frac{\pi}{12})$.

    \item \textbf{2pts} Find $(\frac{1}{2}+\frac{1}{2}i)^{10}$.

\textbf{Solution:}    
    
    First, convert to polar form:
    
    $\frac{1}{2}+\frac{1}{2}i = \frac{\sqrt{2}}{2}(\cos\frac{\pi}{4}+i\sin\frac{\pi}{4})$
    
    Using De Moivre's Theorem:
    
    $(\frac{1}{2}+\frac{1}{2}i)^{10} = (\frac{\sqrt{2}}{2})^{10}(\cos(10\cdot\frac{\pi}{4})+i\sin(10\cdot\frac{\pi}{4}))$
    
    $= (\frac{2^{1/2}}{2})^{10}(\cos(\frac{5\pi}{2})+i\sin(\frac{5\pi}{2}))$
    
    $= (\frac{1}{2^{1/2}})^{10}(\cos(\frac{\pi}{2}+2\pi)+i\sin(\frac{\pi}{2}+2\pi))$
    
    $= (\frac{1}{2^5})(0+i\cdot 1) = \frac{i}{32}$

    \item \textbf{2pts} Evaluate or Simplify:
    \begin{enumerate}
        \item $e^{i\pi}$
        \item $e^{-1+i\pi/2}$
    \end{enumerate}
    
    \textbf{Solution for (a):}
    $e^{i\pi} = \cos\pi+i\sin\pi = -1+i(0) = -1$.
    
    \textbf{Solution for (b):}
    $e^{-1+i\pi/2} = e^{-1}e^{i\pi/2} = e^{-1}\bigg( \cos{\tfrac{\pi}{2}} + i \sin{\tfrac{\pi}{2}} \bigg) = \cfrac{1}{e}\bigg(0 + i(1) \bigg)= \cfrac{i}{e}$

    \item \textbf{2pts} Evaluate the expression below and write your answer in the form $a+ib$.
    \begin{enumerate}
        \item $(5-i6)+(3+i2)$
        \item $\frac{3}{4-i3}$
    \end{enumerate}
    \textbf{Solution for (a):}
    $(5-i6)+(3+i2) = (5+3)+(-6+2)i = 8-4i$.
    
    \textbf{Solution for (b):}
    $\frac{3}{4-i3} = \frac{3}{4-i3} \times \frac{4+i3}{4+i3} = \frac{12+i9}{4^2+3^2} = \frac{12+i9}{16+9} = \frac{12+i9}{25} = \frac{12}{25}+\frac{9}{25}i$.

    \item \textbf{2pts} Find the complex conjugate and modulus of the number:
    \begin{enumerate}
        \item $12+i5$
        \item $-1+2\sqrt{2}i$
    \end{enumerate}
    \textbf{Solution for (a):}
    
    Complex conjugate: $12-i5$.
    
    Modulus: $|12+i5| = \sqrt{12^2+5^2} = \sqrt{144+25} = \sqrt{169} = 13$.
    
    \textbf{Solution for (b):}
    
    Complex conjugate: $-1-2\sqrt{2}i$.
    
    Modulus: $|-1+2\sqrt{2}i| = \sqrt{(-1)^2+(2\sqrt{2})^2} = \sqrt{1+8} = \sqrt{9}=3$.

    \item \textbf{2pts} Apply De Moivre's Theorem to simplify:
    \begin{enumerate}
        \item $(1+i)^{20}$
        \item $(1-\sqrt{3}i)^{5}$
        \item $(1-i)^{8}$
    \end{enumerate}

    \textbf{Solution for (a):}
    
    Convert $1+i$ to polar form: $r=\sqrt{1^2+1^2}=\sqrt{2}$, 
    
    $\theta=\tan^{-1}(\frac{1}{1})=\frac{\pi}{4}$.
    
    $1+i = \sqrt{2}(\cos\frac{\pi}{4}+i\sin\frac{\pi}{4})$.
    
    $(1+i)^{20} = (\sqrt{2})^{20}(\cos(20\cdot\frac{\pi}{4})+i\sin(20\cdot\frac{\pi}{4}))$
   
    $= 2^{10}(\cos(5\pi)+i\sin(5\pi)) = 1024(-1+0i) = -1024$.

    \textbf{Solution for (b):}

    Convert $1-\sqrt{3}i$ to polar form: $r=\sqrt{1^2+(-\sqrt{3})^2}=2$, $\theta=\tan^{-1}(\frac{-\sqrt{3}}{1})=-\frac{\pi}{3}$.
    
    $1-\sqrt{3}i = 2(\cos(-\frac{\pi}{3})+i\sin(-\frac{\pi}{3}))$.

    $(1-\sqrt{3}i)^5 = 2^5(\cos(5\cdot(-\frac{\pi}{3}))+i\sin(5\cdot(-\frac{\pi}{3})))$

    $= 32(\cos(-\frac{5\pi}{3})+i\sin(-\frac{5\pi}{3})) = 32(\cos(\frac{\pi}{3})+i\sin(\frac{\pi}{3}))$

    $= 32(\frac{1}{2}+i\frac{\sqrt{3}}{2}) = 16(1+i\sqrt{3})$.

    \textbf{Solution for (c):}
    
    Convert $1-i$ to polar form: $r=\sqrt{1^2+(-1)^2}=\sqrt{2}$, $\theta=\tan^{-1}(\frac{-1}{1})=-\frac{\pi}{4}$.

    $1-i = \sqrt{2}(\cos(-\frac{\pi}{4})+i\sin(-\frac{\pi}{4}))$.

    $(1-i)^8 = (\sqrt{2})^8(\cos(8\cdot(-\frac{\pi}{4}))+i\sin(8\cdot(-\frac{\pi}{4})))$

    $= 2^4(\cos(-2\pi)+i\sin(-2\pi)) = 16(1+0i)=16$.

    \item \textbf{3pts} Use Euler's formula to prove the following formulas for $\cos x$ and $\sin x$:
    \begin{enumerate}
        \item $\cos x = \frac{e^{ix}+e^{-ix}}{2}$
        \item $\sin x = \frac{e^{ix}-e^{-ix}}{2i}$
    \end{enumerate}
    \textbf{Solution:}
    Euler's formula states:
    $e^{ix} = \cos x + i\sin x$
    $e^{-ix} = \cos x - i\sin x$

    To prove (a), add the two equations:
    $e^{ix}+e^{-ix} = (\cos x + i\sin x) + (\cos x - i\sin x) = 2\cos x$.
    Therefore, $\cos x = \frac{e^{ix}+e^{-ix}}{2}$.

    To prove (b), subtract the second equation from the first:
    $e^{ix}-e^{-ix} = (\cos x + i\sin x) - (\cos x - i\sin x) = 2i\sin x$.
    Therefore, $\sin x = \frac{e^{ix}-e^{-ix}}{2i}$.

\end{enumerate}
\end{document}

